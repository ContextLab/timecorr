\documentclass[12pt]{article}
\usepackage[utf8]{inputenc}
\usepackage{tikz}
\usepackage{amsmath,amsfonts,amssymb,amsthm}
\usepackage[vlined, ruled]{algorithm2e}
\usepackage{geometry}
\usepackage[noend]{algpseudocode}
\usetikzlibrary{bayesnet}
\usepackage[nottoc,numbib]{tocbibind}
\setlength{\parskip}{1em}
\geometry{letterpaper,left=1.5in,right=1in,top=1in,bottom=1in}
\setlength\parindent{0pt}
\linespread{1.5}
\newcommand{\E}{\mathrm{E}}
\newcommand{\Var}{\mathrm{Var}}
\newcommand{\N}{\mathcal{N}}
\newcommand{\tr}{tr}
\begin{document}
\subsection{ISFC}
The ISFC is a process through which we find the stimulus-dependent activations in our fMRI dataset by cross referencing and averaging data from multiple subjects.\\
\begin{enumerate}
\item For each subject s, we find the average activation of all other subjects:
\begin{align*}
O_s=\frac{\sum_{i\neq s}^N S_i}{N-1}
\end{align*}
where $S_i$ represents the activation matrix for subject $i$ and $N$ represents the total number of subjects.
\item Find the correlation matrices between the voxel activations for each subject $S_i$ and the average voxel activations of all other subjects $O_{i}$ using the timecorr method with variance $\sigma$. To find the correlation between voxel activation $S^i_t$ of subject $S$ for voxel $i$ at time $t$ and voxel activation $O^j_t$ of the average of other subjects for voxel $j$ at timepoint $t$ is obtained through the following equation:
\begin{align*}
C(S^i_t,O^j_t) = \frac{1}{Z}\frac{\sum_{l=0}^T (S_l^i - \bar{S^i_t})\cdot(O^j_l - \bar{O^j_t})\cdot \mathcal{N}(l|t,\sigma)}{\sigma_{S_t^i} \cdot \sigma_{O_t^j}}
\end{align*}
Where
\begin{align*}
Z &= \sum_{l=0}^T \mathcal{N}(l|t,\sigma)\\
\bar{S^i_t} &=\frac{1}{Z} \sum_{l=0}^T S^i_l \cdot \mathcal{N}(l|t,\sigma)\\
\bar{O^i_t} &=\frac{1}{Z} \sum_{l=0}^T O^i_l \cdot \mathcal{N}(l|t,\sigma)\\
\sigma_{S_t^i} &=\sqrt{ \frac{1}{Z}\sum_{l=0}^T (S_l^i-\bar{S_t^i})^2 \cdot \mathcal{N}(l|t,\sigma)}\\
\sigma_{O_t^i} &=\sqrt{ \frac{1}{Z}\sum_{l=0}^T (O_l^i-\bar{O_t^i})^2 \cdot \mathcal{N}(l|t,\sigma)}\\
\end{align*}
\item Apply Fisher Z-transformation to every element $r$ of the correlation matrices for each subject at each time points to obtain the corresponding Z-correlation matrices:
\begin{align*}
z = \frac{1}{2}\ln(\frac{1+r}{1-r})
\end{align*}
\item Average the Z-correlation matrices $Z_i$ across all subjects:
\begin{align*}
S_Z = \frac{1}{N}\sum^N_{i=1}Z_i
\end{align*}
\item Apply inverse Z-transformation to the average Z-correlation matrix to obtain the Inter-subject Functional Connectivity (ISFC) mean correlation matrix:
\begin{align*}
ISFC = \frac{\exp(S_Z+S_Z^T)-1}{\exp(S_Z+S_Z^T)+1}
\end{align*}
\end{enumerate}
\end{document}
